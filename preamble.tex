\usepackage{amsmath}
\usepackage{graphicx}
\usepackage{subcaption}
\usepackage{amsfonts}
\usepackage{amssymb}
\usepackage{hyperref}
\usepackage{booktabs}
\usepackage{fontawesome5}

\usepackage{siunitx}
\usepackage{mathtools}

\usepackage{float}
\usepackage{tikz}
\usepackage{tikz-3dplot}
\usetikzlibrary{calc, decorations.markings}
\usetikzlibrary{plotmarks}
\usepackage{pgfplots}
\usepgfplotslibrary{fillbetween}

\tdplotsetmaincoords{60}{110}
%\tdplotsetmaincoords{90}{0}
\pgfmathsetmacro{\radius}{1}
\pgfmathsetmacro{\thetavec}{0}
\pgfmathsetmacro{\phivec}{0}
\pgfmathsetmacro{\dz}{.2}
\pgfmathsetmacro{\dzh}{\dz*.5}
\pgfmathsetmacro{\n}{4}

\pgfdeclareplotmark{gradient}{
    \fill [
        draw=black,
        left color=white,
        right color=black,
        shading angle=45
    ] (0,0) circle [radius=\pgfplotmarksize];
}


\newcommand\notebook[1]{\noindent\faLaptopCode{} -- \textbf{Dans le notebook} : #1}
\newcommand\questions{\noindent\faQuestionCircle[regular] -- \textbf{Questions} :}
